% !TeX root = ./main.tex

\ustcsetup{
  title              = {张量网络优化算法},
  title*             = {Optimization Algorithms of \\Tensor Network},
  author             = {孙玉鑫},
  author*            = {Sun Yuxin},
  speciality         = {凝聚态物理},
  speciality*        = {Condensed Matter Physics},
  supervisor         = {何力新~教授},
  supervisor*        = {Prof. He Lixin},
  % date               = {2017-05-01},  % 默认为今日
  % professional-type  = {专业学位类型},
  % professional-type* = {Professional degree type},
  % secret-level       = {秘密},     % 绝密|机密|秘密,注释本行则不保密
  % secret-level*      = {Secret},  % Top secret|Highly secret|Secret
  % secret-year        = {10},      % 保密年限
}


% 加载宏包
\usepackage{graphicx}
\usepackage{booktabs}
\usepackage{longtable}
\usepackage[ruled,linesnumbered]{algorithm2e}
\usepackage{siunitx}
\usepackage{amsthm}
\usepackage{mathtools}
\DeclarePairedDelimiter\bra{\langle}{\rvert}
\DeclarePairedDelimiter\ket{\lvert}{\rangle}
\DeclarePairedDelimiterX\braket[2]{\langle}{\rangle}{#1 \delimsize\vert #2}

% 数学命令
\input{math-commands.tex}

% 配置图片的默认目录
\graphicspath{{figures/}}

% 用于写文档的命令
\DeclareRobustCommand\cs[1]{\texttt{\char`\\#1}}
\DeclareRobustCommand\pkg{\textsf}
\DeclareRobustCommand\file{\nolinkurl}


% hyperref 宏包在最后调用
\usepackage{hyperref}
