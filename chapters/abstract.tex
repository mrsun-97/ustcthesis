% !TeX root = ../main.tex

\ustcsetup{
  keywords = {
    张量网络; 矩阵乘积态; 投影纠缠对; 梯度优化算法; 方差缩减的随机梯度下降
  },
  keywords* = {
    Tensor Network; Matrix Product State; Projected Entangled Pair State; Gradient Optimization; 
    Stochastic Variance Reduced Gradient
  },
}

\begin{abstract}
  张量网络正成为量子多体系统研究中备受欢迎的表示方法,它的出现给强关联系统的研究提供了全新的视角。
  
  本文从基本的张量网络定义开始,简要地介绍了张量网络的常用形式MPS与PEPS,并通过MPS介绍张量网络中的时间演化算法,以及进一步衍生的虚时演化求基态算法。之后本文详细介绍了张量网络的梯度优化算法以及与随机梯度下降有关的工作,并尝试将机器学习中优化梯度算法的SVRG方法使用在张量网络上以提高性能。
  
\end{abstract}

\begin{abstract*}
  The Tensor Network method, which gives people a new approach to Strongly Correlated Systems, is becoming more and more popular in quantum many body research.
  
  This paper is a brief introduction from the definition of tensor network to its commonly used form: MPS and PEPS. We briefly summarize the Time Evolution methods in MPS and the imaginary time evolution technique. Besides, we illustrate the procedure of gradient optimization and try to improve its performance with some optimization algorithms in machine learning.

\end{abstract*}
