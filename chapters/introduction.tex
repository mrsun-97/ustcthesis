% !TeX root = ../main.tex

\chapter{绪论}

\subsection{背景介绍}

理解量子多体系统也许是凝聚态物理中最具挑战性的问题。科学家们一直在致力于研究高温超导,拓扑序,量子自旋液体等现象背后的物理本质,但仅仅通过理论分析已经很难得出这种复杂系统的所有性质,因此人们提出简化后的理论模型,并通过计算机模拟来尝试复现真实世界中存在的现象。

近些年来,张量网络作为一种数值模拟手段正在受到越来越多人的重视。不同于传统的给出量子态各个基矢上的系数的表示方法,张量网络将量子态表示为一系列张量的缩并,其相比于经典的表示方法,天然的反映出量子系统中的纠缠与关联的特性。

张量网络表示法具有非常高的灵活性,我们可以研究一个系统在不同维度下的表现,可以研究有限系统与无穷大系统之间的区别,还可以研究系统在不同边界条件、不同对称性下的性质,也可以研究波色、费米或分数系统的特点。并且在张量网络上发现的技术在其他的领域也有很好的应用空间,如量子化学等。

\subsection{为何要使用张量网络}

\subsubsection{扩展技术的边界}

目前所有的模拟方法都有其自身的局限性。最常见的方法如平均场理论\cite{vedralMeanfieldApproximationsMultipartite2004},它无法很好的描述量子关联;量子蒙特卡洛方法\cite{QMC1999}对于费米子系统等不适用,会出现符号问题;精确对角化方法\cite{2012lanczos}计算量过大,这使得可计算的系统只能被限制在很小的区域;密度泛函理论\cite{dengDensityFunctionalTheory2014}依赖于我们对电子交换关联泛函的选择,且通常针对不同的问题,不同的泛函形式各有优劣。同样的,张量网络算法并不完美,它也有一定的适用范围。它对量子态的纠缠形式做出了限制,但是也和其他方法一样,给我们提供全新的视角来模拟量子态,使得更多的模型可被计算机模拟。

\subsubsection{新的物理描述}

通过彼此相连的张量构成的网络来表示量子态,这种方法抓住了系统的主要特征——纠缠。通过这种方式来描述量子态与经典方式截然不同,我们不再是仅仅处理一堆基矢上的系数,而是通过图表的形式描述与理解量子体系。它被认为是描述量子系统的自然语言,对凝聚态物理来说是一种全新的语言。

\subsubsection{真实态空间过大}

使得张量网络备受欢迎的一个主要原因可能便是真实系统的态空间往往过于巨大了。多体系统被表示为多个系统的直积,表示系统需要的基矢将随系统规模指数及上升。即使是最简单的$N$个自旋-$\frac12$自旋,其态空间维度也是$2^N$。

幸运的是我们没有必要将所有基矢都包括进来,因为绝大多数自然界存在的多体系统,其相互作用都是局域化的。这使得一些系统的基态满足纠缠熵面积定律,我们将在下文中介绍这一性质。我们通常不必考虑整个希尔伯特空间,而只需考虑满足纠缠熵面积定律的状态,这种量子态通常只占希尔伯特空间之一耦,却恰恰是张量网络擅长表示的。
