% !TeX root = ../main.tex

\begin{acknowledgements}

我非常感谢在大学生活中帮助我,指导我的老师与同学们。

首先我要感谢实验室的何力新教授对我科研方面的指导。作为本科生初次接触科研工作,有很多方面都不了解,而对量子体系的模拟计算由于需要大量课本外的知识,对本科生来说更是难上加难。我十分感谢何老师给我在他的实验室学习的机会,并且在研究方向上给予指导,让我从简单问题入手循序渐进,逐渐了解实验室研究所需的各种知识;以及在我论文写作遇到困难时,何老师也认真指导我如何克服困难解决问题,如何正确的安排论文结构等。

感谢实验室中已经毕业出国的李东昊师兄对我的帮助。师兄作为只大我一届的学长,与我一个实验室,也曾是我所在的学生会部门的部长,因此我在很多事情上都与他有共同语言。当我在学习生活中遇到问题时,我也更倾向于找李师兄一起讨论。另外我能够了解到并最终加入何老师的实验室,也是李东昊师兄在听到我希望做计算物理相关工作时向我推荐的。

感谢这两年以来,实验室中的其他老师与师兄们的帮助。感谢董少钧师兄在我初到实验室时对我的辅导,每次我遇到各种问题时,师兄总是认真的给我讲解,直到我彻底明白之前不懂的知识点,让我的学习效率远超过自己去阅读晦涩难懂的论文。师兄就像是我在实验室的辅导员一样,辅导着我科研的方方面面。感谢张浩师兄对我的指导,师兄用Linux系统工作已达到炉火纯青的地步,并且在系统与编程方面向我分享了许多有用的知识,激发我进一步学习的兴趣。感谢实验室中的韩永建老师,王超师兄与张盟师兄,他们在与我讨论问题时能认真的向我讲解我的知识盲区,在我组会报告出现错误时敏锐的找出错误并纠正,对我的科研与学习给予方方面面的帮助。

特别感谢我的班主任封常青老师对我学习生活中的帮助。封老师是一位对待学生学习生活问题非常负责的老师,每次同学们遇到问题时封老师都会抽出自己的时间帮助同学们解决问题;平时发布学校通知,开班会,找同学谈话了解情况等辅导班级的工作,封老师也非常上心。本论文在写作的过程中,我的家人生病住院,我也因此心态受到影响,耽误了论文的写作进度。封老师察觉到我的异常后,便迅速找我了解情况,并在后续不间断的关心、询问我的情况,督促我完成论文,在我浑浑噩噩整日无所事事时,帮助我排忧解难,点醒颓废的我。没有封老师的帮助,我可能就会因为自己的颓废而使得各种问题接踵,最终影响自己的毕业。现在我想,毕业多年之后,我仍然会对自己能幸运地被分到封老师的班级而心存感激。

\end{acknowledgements}
