% !TeX root = ../main.tex

\chapter{张量网络态}

\section{张量}

\subsection{张量的定义}

\subsubsection{抽象定义}

若$V_i$与$W_i$都是域$\mathbb{F}$上的有限线性空间,$\otimes$是张量积,那么一个张量即是原空间$W_i$与对偶空间$V^*_i$的张量积空间中的一个元素,若张量积空间由$p$个原空间与$q$个对偶空间构成,则其上张量被称为$(p,q)$阶的:
\[
\text{order-}(p,q) \text{ tensor } T \in W_1 \otimes W_2 \otimes \dots \otimes W_p 
\otimes V^*_1 \otimes V^*_2 \otimes \dots \otimes V^*_q
\]

若$\{{e^{(i)}}_k\}$与$\{{\eta^{(i)}}_k\}$分别为空间$W_i$与对偶空间$V^*_i$的基,则利用爱因斯坦求和约定,张量可以被表示为分量和的形式:
\[
T=T^{i_1\dots i_p}_{j_1\dots j_q} \,\, {e^{(1)}}_{i_1}\otimes\dots\otimes{e^{(p)}}_{i_p}
\otimes {\eta^{(1)j_1}}\otimes\dots\otimes{\eta^{(q)j_q}}
\]

用于描述量子多体系统的张量网络是由具体的张量表示的,因此为了更方便与准确的表示本文中的张量,我们将张量的定义特殊化以方便描述本文中使用的张量。

\subsubsection{具体定义}

一个实数域或复数域上的有限多维数组$\{T_{i_1 i_2\dots i_p}\}$被称为一个张量,若表示一个张量的数组的维度为$p$,则称此张量为$p$阶的。若第$k$个指标$i_k$的取值范围为集合$S_{k}$,则$\lvert S_{k}\rvert$称为指标$i_k$的维数。若两个同阶张量的指标维数也依次相等,则称两个张量是相同形状的。

相较于上文中的定义,这里应用了以下改动:
\paragraph{指定数域}
一个量子系统至多在复数域即可被表示,因此这里把数域指定为实数域或复数域,此时张量可表示为实数或复数域上的多维数组。
\paragraph{约定形式}
本文中的张量有两种指标,一种是由实空间取向标记的缩并指标,一种是由对应的自旋分量(通常为$S_z$的本征值)标记的物理指标。这里我们至多只需研究张量在不同自旋基矢下的变换,因此可以不区分协变与逆变指标,默认张量为$(0,p)$或$(p,0)$阶的。

\subsection{张量的运算}

\subsubsection{加法运算}

两个相同形状的张量$A$与$B$之间可以定义加法:$C = A+B$,即:
\begin{equation}
C_{i_1 i_2 \dots i_p} = A_{i_1 i_2 \dots i_p} + B_{i_1 i_2 \dots i_p}
\end{equation}


\subsubsection{数乘运算}

对任意$\alpha \in \mathbb{C}$,可定义张量与其的数乘运算:$M' = \alpha M = M \alpha$,即:
\begin{equation}
M_{i_1 i_2 \dots i_p}' = \alpha M_{i_1 i_2 \dots i_p}
\end{equation}

\subsubsection{缩并运算}

若两个张量$A$与$B$间的一个或者一些指标维数分别相同,则可以定义缩并运算对重复指标遍历求和:
\begin{equation}
C_{\alpha\beta} = \sum_{\gamma=1}^{D_\gamma} A_{\alpha\gamma} B_{\gamma\beta}
\end{equation}
或
\begin{equation}
C_{\alpha\beta} = \sum_{\delta,\gamma,\nu=1}^{D_\delta,D_\gamma,D_\nu} A_{\alpha\delta\gamma\nu} B_{\gamma\beta\nu\delta}
\end{equation}
所得张量的未缩并指标可继续缩并:
\begin{equation}
\label{eq.1.5}
M_{\alpha m n} = \sum_{\beta=1}^{D_\beta} C_{\alpha\beta} F_{\beta m n}
	= \sum_{\delta,\gamma,\nu,\beta=1}^{D_\delta,D_\gamma,D_\nu,D_\beta} A_{\alpha\delta\gamma\nu} B_{\gamma\beta\nu\delta} F_{\beta m n}
\end{equation}

\subsubsection{分解运算}

分解可看作缩并的逆运算。若给定张量$\left\{M_{\alpha m n}\right\}$,则如公式$(\ref{eq.1.5})$中将张量表示为几个张量的缩并的形式,称为张量的分解。\\

由以上定义易知,相同形状的张量构成线性空间,其加法与数乘的性质与矢量一致,且张量之间缩并的先后顺序不影响结果。下文中为了书写方便,若无特殊说明,张量缩并均采用爱因斯坦求和规则,即默认重复出现的指标需要求和。例如公式$(\ref{eq.1.5})$重新写为:
\[
M_{\alpha m n} = C_{\alpha\beta} F_{\beta m n}
	= A_{\alpha\delta\gamma\nu} B_{\gamma\beta\nu\delta} F_{\beta m n}
\]


\section{张量网络}

\subsection{张量网络的定义}

张量网络是一组张量的集合与其上的缩并规则构成的数学结构。整个张量网络等效于一个巨大的张量\cite{hsiaoJournalClubBrief}。

张量网络的全部结构由无向图$G=(V,E)$表示,其中顶点集$V$为空顶点集$V_L$与张量顶点$V_T$的不交并,边集$E$为$E_L$与$E_T$的不交并。表示规则为:
\paragraph{每个张量与$V_T$中元素一一对应;}

\paragraph{每个未缩并指标与$V_L$,$E_L$中元素都一一对应;}

\paragraph{每对缩并指标与$E_T$中元素一一对应。}

边集$E$中元素被称为脚,$E_L$中元素只与一个张量顶点相连,我们称其为“悬空脚”;$E_T$中元素连接两个张量顶点,我们称其为“缩并脚”或者“键”。

\subsection{张量网络的图表示法}

每个张量网络都对应一个无向图,因此我们可以将图画出,通过图形来了解张量网络的结构。在通常的无向图上,我们将张量顶点扩大为空心图形着重显示,用以代表此顶点对应点的张量。这种无向图的表示被称为张量网络图。

在张量网络图中,每个$n$阶张量一定有$n$个脚,需要缩并的张量之间公共脚的个数对应需要缩并的指标个数,每个脚对应的指标的维数不表示在图中。具体的表示如图 -----------

相比于文字叙述或代数表达,张量网络图能够更直观的表示张量网络的结构,也使得我们能更高效的描述对网络施加的缩并、分解等操作。

\subsection{张量网络的运算}

\subsubsection{张量网络的缩并}

一个张量网络中的张量之间存在缩并关系,当我们应用这些关系将某些张量缩并为一个张量时,其张量网络图中对应的边表现为收缩至消失,且其两端的张量顶点融合为一个,如图-------。这种操作被称之为张量网络对其边(键)的缩并。

\subsection{张量网络的分解}

如同张量的分解可看做张量缩并的逆运算,张量网络的分解也可借助其缩并来定义。当我们将网络中的某些张量分别拆分为其他张量的缩并时,这种操作被称之为张亮的分解,如图——————。

\section{张量网络与量子态}

\subsection{量子态的表示}

上文主要介绍了张量网络相关的性质,本节我们将说明张量网络具体应怎样应用至量子态的表示中。

通常情况下,我们会遇到由$N$粒子构成的量子多体系统,其每一个粒子态空间中有$p$个态。最常见的例子即是一个由自旋$\frac12$的粒子组成的$N$粒子自旋系统,表示这样的量子态的态矢$\ket{\varphi}\in\left(\mathbb{C^d}\right)^{\otimes N}$为
\begin{equation}
\ket{\varphi} = \sum_{j_1 \dots j_N} c_{j_i \dots j_N}\ket{j_1 \dots j_N}
	= \sum_{j_1 \dots j_N}c_{j_i \dots j_N}\ket{j_1}\otimes \dots \otimes\ket{j_N}
\end{equation}
其系数$c_{j_i \dots j_N}\in\mathbb{C}$对于所有的指标构成一个张量 $\left\{c_{j_i \dots j_N}\right\}$,将其用张量网络图表示如图————————。
\paragraph{四级节标题}

\subparagraph{五级节标题}

abcde fghijk lmn

\section{脚注}

Lorem ipsum dolor sit amet, consectetur adipiscing elit, sed do eiusmod tempor
incididunt 
\footnote{Ut enim ad minim veniam, quis nostrud exercitation ullamco laboris
  nisi ut aliquip ex ea commodo consequat.
  Duis aute irure dolor in reprehenderit in voluptate velit esse cillum dolore
  eu fugiat nulla pariatur.}
