% !TeX root = ../main.tex

\chapter{张量网络态}

\section{张量}

\subsection{张量的定义}

\subsubsection{抽象定义}

若$V_i$与$W_i$都是域$\mathbb{F}$上的有限线性空间,$\otimes$是张量积,那么一个张量即是原空间$W_i$与对偶空间$V^*_i$的张量积空间中的一个元素,若张量积空间由$p$个原空间与$q$个对偶空间构成,则其上张量被称为$(p,q)$阶的:
\[
\text{order-}(p,q) \text{ tensor } T \in W_1 \otimes W_2 \otimes \dots \otimes W_p 
\otimes V^*_1 \otimes V^*_2 \otimes \dots \otimes V^*_q
\]

若$\{{e^{(i)}}_k\}$与$\{{\eta^{(i)}}_k\}$分别为空间$W_i$与对偶空间$V^*_i$的基,则利用爱因斯坦求和约定,张量可以被表示为分量和的形式:
\[
T=T^{i_1\dots i_p}_{j_1\dots j_q} \,\, {e^{(1)}}_{i_1}\otimes\dots\otimes{e^{(p)}}_{i_p}
\otimes {\eta^{(1)j_1}}\otimes\dots\otimes{\eta^{(q)j_q}}
\]

用于描述量子多体系统的张量网络是由具体的张量表示的,因此为了更方便与准确的表示本文中的张量,我们将张量的定义特殊化以方便描述本文中使用的张量。

\subsubsection{具体定义}

一个实数域或复数域上的有限多维数组$\{T_{i_1 i_2\dots i_p}\}$被称为一个张量,若表示一个张量的数组的维度为$p$,则称此张量为$p$阶的。若第$k$个指标$i_k$的取值范围为集合$S_{k}$,则$\lvert S_{k}\rvert$称为指标$i_k$的维数。若两个同阶张量的指标维数也依次相等,则称两个张量是相同形状的。

相较于上文中的定义,这里应用了以下改动:
\paragraph{指定数域}
一个量子系统至多在复数域即可被表示,因此这里把数域指定为实数域或复数域,此时张量可表示为实数或复数域上的多维数组。
\paragraph{约定形式}
张量网络中的张量有两种指标,一种是由实空间指向标记的缩并指标,一种是由对应的自旋分量(通常为$S_z$的本征值)标记的物理指标。这里我们至多只需研究张量在不同自旋基矢下的变换,因此可以不区分协变与逆变指标,默认张量为$(0,p)$或$(p,0)$阶的。

\subsection{张量的运算}

\subsubsection{加法运算}

两个相同形状的张量$A$与$B$之间可以定义加法:$C = A+B$,即:
\[
C_{i_1 i_2 \dots i_p} = A_{i_1 i_2 \dots i_p} + B_{i_1 i_2 \dots i_p}
\]


\subsubsection{数乘运算}

对任意$\alpha \in \mathbb{C}$,可定义张量与其的数乘运算:$M' = \alpha M = M \alpha$,即:
\[
M_{i_1 i_2 \dots i_p}' = \alpha M_{i_1 i_2 \dots i_p}
\]

\subsubsection{缩并运算}

若两个张量$A$与$B$间的一个或者一些指标维数分别相同,则可以定义缩并运算对重复指标遍历求和:
\[
C_{\alpha\beta} = \sum_{\gamma=1}^{D_\gamma} A_{\alpha\gamma} B_{\gamma\beta}
\]
或
\[
C_{\alpha\beta} = \sum_{\delta,\gamma,\nu=1}^{D_\delta,D_\gamma,D_\nu} A_{\alpha\delta\gamma\nu} B_{\gamma\beta\nu\delta}
\]

由以上定义易知,相同形状的张量构成线性空间,其加法与数乘的性质与矢量一致,且两个张量之间缩并的顺序不影响结果。

\section{张量网络}

\subsection{张量网络的定义}

张量网络是一组张量的集合与其上的缩并规则构成的数学结构。

张量网络的全部结构由无向图$G=(V,E)$表示,其中顶点集$V$为空顶点集$V_L$与张量顶点$V_T$的不交并,边集$E$为$E_L$与$E_T$的不交并。表示规则为:
\paragraph{每个张量与$V_T$中元素一一对应;}

\paragraph{每个未缩并指标与$V_L$,$E_L$中元素都一一对应;}

\paragraph{每对缩并指标与$E_T$中元素一一对应。}

由上面的表示,每个张量的阶数与其对应$v\in V_T$的节点度相等;若两个张量间有$k$个缩并指标,其对应顶点间就有$k$个公共边。

\paragraph{四级节标题}

\subparagraph{五级节标题}

abcde fghijk lmn

\section{脚注}

Lorem ipsum dolor sit amet, consectetur adipiscing elit, sed do eiusmod tempor
incididunt 
\footnote{Ut enim ad minim veniam, quis nostrud exercitation ullamco laboris
  nisi ut aliquip ex ea commodo consequat.
  Duis aute irure dolor in reprehenderit in voluptate velit esse cillum dolore
  eu fugiat nulla pariatur.}
